\subsection{Phân loại}
Để thuận lời cho việc phân loại một cách dễ dàng và giữ cho phân loại các kĩ thuật trực quan hóa đơn giản, chúng tôi tập trung vào những khía cạnh cốt lõi của dữ liệu, thời gian và trực quan hóa. Qua nhiều ví dụ trực quan hóa, chúng tôi sẽ minh họa những khả năng ứng dụng của chúng và những tính năng của các kĩ thuật trực quan hóa này.
\begin{itemize}
    \item \textbf{Dữ liệu} 
    \item[] \begin{itemize}
        \item \textit{Khung tham chiếu} - trừu tượng, không gian
        \item \textit{Các biến} - đơn biến, đa biến
    \end{itemize}
    \item \textbf{Thời gian}
    \item[] \begin{itemize}
        \item \textit{Sắp xếp} - tuyến tính, tuần hoàn
        \item \textit{Thời gian nguyên thủy} - thời điểm, khoảng
    \end{itemize}
    \item \textbf{Trực quan hóa}
    \item[] \begin{itemize}
        \item \textit{Ánh xạ} - tĩnh, động
        \item \textit{Số chiều} - 2D, 3D
    \end{itemize}
\end{itemize}
Để thiết lập những kĩ thuật trực quan hóa khác nhau cho dữ liệu hướng thời gian, chúng ta cần trả lời những câu hỏi sau:
\begin{itemize}
    \item \textit{Cái gì} được biểu diễn ? \textit{Thời gian và dữ liệu}
    \item \textit{Tại sao} nó được biểu diễn ? \textit{Tác vụ của người dùng}
    \item Nó được biểu diễn \textit{thế nào} ? \textit{Biểu diễn trực quan}
\end{itemize}
