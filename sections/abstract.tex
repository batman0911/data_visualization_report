Trong chương này chúng tôi giải thích chi tiết các kĩ thuật đồ họa cho dữ liệu hướng thời gian. Chúng tôi tổ chức và cấu trúc chương theo cuốn sách ... Do đó, trước tiên chúng tôi nhấn mạnh tầm quan trọng của việc xử lý các yếu tố thời gian qua ví dụ. Thứ hai, chúng tôi đưa ra các khái niệm cần thiết và các khía cạnh của thời gian và dữ liệu hướng thời gian. Các tập dữ liệu cơ bản đã có trong chương 2. Ở chương này, chúng tôi tập trung vào các đặc điểm của thời gian. Thứ 3, chúng tôi cung cấp một cái nhìn tổng quan về các kĩ thuật trực quan hóa khác nhau. Thứ tư, chúng tôi giải thích ngắn gọn về  \textit{TimeBench}, là mô hình và một thư viện phần mềm phân tích dữ liệu hướng thời gian. Chúng tôi kết luận với một đánh giá về phân. Chúng tôi kết luận với đánh giá phân loại theo các kĩ thuật trực quan hóa được trình bày và giới thiệu \textit{TimeViz Browser}, là một kho lưu trữ để hỗ trợ các nhà nghiên cứu và học viên trong việc tìm kiếm các kĩ thuật trực quan hóa tích hợp, tiếp theo là các bài đọc, bài tập và đồ án thực hành.