\subsection{Các đặc tính của dữ liệu hướng thời gian}
Cũng như các tính chất của thời gian, dữ liệu có tác động lớn đến việc thiết kế phương pháp trực quan hóa. Chúng ta cùng lược qua những yếu tố chính của dữ liệu liên quan đến thời gian.
\begin{itemize}
    \item \textit{Quy mô}: \textit{định tính}, \textit{định lượng}. Dữ liệu định lượng được dựa trên một thang đo (rời rạc hoặc liên tục). Dữ liệu định tính mô tả tập hợp các phần tử dữ liệu có thứ thự hoặc không có thứ tự.
    \item \textit{Hệ quy chiếu}: \textit{trừu tượng}, \textit{không gian}. Dữ liệu trừu tượng (ví dụ tài khoản ngân hàng) được thu thập trong bối cảnh phi không gian. Dữ liệu không gian (ví dụ dữ liệu điều tra dân số) chứa thông tin về không gian ví dụ như vị trí địa lý.
    \item \textit{Loại dữ liệu}: \textit{sự kiện}, \textit{trạng thái}. Sự kiện có thể được hiểu là các điểm đánh dấu sự thay đổi trạng thái (ví dụ sự khời hành của đoàn tàu), trong khi đó trạng thái đặc trưng cho các giai đoạn liên tục giữa các sự kiện (ví dụ đoàn tàu đang chạy trên đường sắt).
    \item \textit{Số biến}: \textit{đơn biến}, \textit{đa biến}. Dữ liệu đơn biến chỉ chứa một giá trị dữ liệu tại một thời điểm, trong trường hợp đa biến, dữ liệu tại mỗi thời điểm thể hiện bằng nhiều giá trị dữ liệu.
\end{itemize}
Những phạm trù cơ bản này tạo thành cơ sở cho việc lựa chọn, thiết lập kĩ thuật trực quan hóa phù hợp cho dữ liệu hướng thời gian. 