\section{Trực quan hóa dữ liệu hướng thời gian}
Dữ liệu và thời gian đã được trình bày trong phần trước, tuy nhiên chúng ta cũng phải xem xét các vấn đề về thiết kế ở cấp độ biểu diễn trực quan. Để biểu diễn sự phụ thuộc vào thời gian của dữ liệu, ta cần đưa vào trục thời gian. Sự đa dạng của các kĩ thuật trực quan hóa bao gồm các cách tiếp cận rất khác nhau. Để tóm lược sự đa dạng này, chúng tôi tập trung vào 2 tiêu chí cơ bản:
\begin{itemize}
    \item \textit{Ánh xạ của thời gian}. Có hai lựa chọn cho ánh xạ thời gian: ánh xạ thời gian vào không gian và ánh xạ từ thời gian vào thời gian. Khi nói đến ánh xạ từ thời gian vào không gian, điều đó có nghĩa là thời gian và dữ liệu được thể hiện trong một hình ảnh duy nhất. Biểu diễn này không tự thay đổi theo thời gian, đó là lý do tại sao chúng tôi gọi đó là trực quan hóa dữ liệu hướng thời gian \textit{tĩnh}. Ngược lại biểu diễn \textit{động} sử dụng thời gian vật lý để truyền đạt sự phụ thuộc vào thời gian của dữ liệu, nghĩa là thời gian ánh xạ vào thời gian. Các kết quả này trong biểu diễn tự thay đổi theo thời gian (ví dụ bản trình chiếu hay hiệu ứng hoạt họa). Lưu ý rằng, có hay không sự tương tác để điều hướng thời gian không ảnh hưởng đến việc cách tiếp cận trực quan hóa là tĩnh hay động.
    \item \textit{Kích thước của không gian biểu diễn}. Chúng ta có thể phân biệt giữa biểu diễn 2D và 3D của dữ liệu hướng thời gian. Cách trực quan hóa sử dụng không gian 2D phải đảm bảo nhấn mạnh trục thời gian vì chiều thời gian và dữ liệu thường có chung biểu diễn sẵn có. Trong trường hợp biểu diễn 3D, chiều hiển thị thứ 3 được đưa vào. Thực tế, nhiều kĩ thuật sử dụng nó như một chiều dành riêng cho trục thời gian, tách thời gian ra khỏi các chiều dữ liệu khác một cách rõ ràng.
\end{itemize}
\subsection{Phân loại}
Để thuận lời cho việc phân loại một cách dễ dàng và giữ cho phân loại các kĩ thuật trực quan hóa đơn giản, chúng tôi tập trung vào những khía cạnh cốt lõi của dữ liệu, thời gian và trực quan hóa. Qua nhiều ví dụ trực quan hóa, chúng tôi sẽ minh họa những khả năng ứng dụng của chúng và những tính năng của các kĩ thuật trực quan hóa này.
\begin{itemize}
    \item \textbf{Dữ liệu} 
    \item[] \begin{itemize}
        \item \textit{Khung tham chiếu} - trừu tượng, không gian
        \item \textit{Các biến} - đơn biến, đa biến
    \end{itemize}
    \item \textbf{Thời gian}
    \item[] \begin{itemize}
        \item \textit{Sắp xếp} - tuyến tính, tuần hoàn
        \item \textit{Thời gian nguyên thủy} - thời điểm, khoảng
    \end{itemize}
    \item \textbf{Trực quan hóa}
    \item[] \begin{itemize}
        \item \textit{Ánh xạ} - tĩnh, động
        \item \textit{Số chiều} - 2D, 3D
    \end{itemize}
\end{itemize}
Để thiết lập những kĩ thuật trực quan hóa khác nhau cho dữ liệu hướng thời gian, chúng ta cần trả lời những câu hỏi sau:
\begin{itemize}
    \item \textit{Cái gì} được biểu diễn ? \textit{Thời gian và dữ liệu}
    \item \textit{Tại sao} nó được biểu diễn ? \textit{Tác vụ của người dùng}
    \item Nó được biểu diễn \textit{thế nào} ? \textit{Biểu diễn trực quan}
\end{itemize}
