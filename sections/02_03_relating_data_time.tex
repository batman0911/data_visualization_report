\subsection{Dữ liệu liên quan và thời gian}
Các khía cạnh liên quan của dữ liệu phụ thuộc vào thời gian đã được kiểm tra rộng rãi trong lĩnh vực cơ sở dữ liệu về thời gian [274,397]. Ở đây, chúng tôi điều chỉnh và phát triển các định nghĩa trong lĩnh vực này. Theo đó, bất kì tập dữ liệu nào cũng có liên quan đến 2 miền thời gian: (1) thời gian bên trong và (2) thời gian bên ngoài. 
\\ \\
\textit{Thời gian bên trong} được coi là chiều thời gian vốn có của mô hình dữ liệu. Thời gian bên trong mô tả thời điểm thông tin tồn tại trong tập dữ liệu là hợp lệ. Ngược lại \textit{thời gian bên ngoài} là thời gian nằm ngoài mô hình dữ liệu. Thời gian bên ngoài là cần thiết để mô tả cách một tập dữ liệu phát triển theo thời gian. Dựa vào số biến thời gian nguyên thủy của thời gian bên trong và thời gian bên ngoài, dữ liệu liên quan đến thời gian có thể được phân loại như sau:
\begin{itemize}
    \item \textit{Dữ liệu tĩnh phi thời gian}. Nếu cả thời gian bên trong lẫn bên ngoài bao gồm một yếu tố thời gian thì dữ liệu hoàn toàn không phụ thuộc vào thời gian. Trong chương này chúng ta sẽ không nghiên cứu kiểu dữ liệu đó.
    \item \textit{Dữ liệu thời gian tĩnh}. Nếu thời gian bên trong chứa nhiều hơn một yếu tố thời gian nguyên thủy, trong khi thời gian bên ngoài chỉ chứa một thì dữ liệu có thể được xem là phụ thuộc vào thời gian. Vì các giá trị được lưu trong dữ liệu phụ thuộc vào thời gian bên trong, dữ liệu thời gian tĩnh có thể được hiểu là một cái nhìn lịch sử về cách một số mô hình xem xét các yếu tố khác nhau của thời gian trong. Chuỗi thời gian chung là một ví dụ tiêu biểu về dữ liệu thời gian tĩnh. 
    \item \textit{Dữ liệu động phi thời gian}. Nếu thời gian bên trong chỉ chứa một mà thời gian bên ngoài chứa nhiều yếu tố thời gian nguyên thủy thì dữ liệu phụ thuộc vào thời gian bên ngoài. Để cho dễ hiểu, dữ liệu thay đổi qua thời gian thì chúng được gọi là động. Vì thời gian bên trong không được xem xét nên chỉ có trạng thái hiện tại của dữ liệu là bảo toàn. Cái nhìn quá khứ không được duy trì ở đây nữa. Có vài kĩ thuật trực quan hóa sẵn có tập trung vào dữ liệu động phi thời gian, tuy nhiên do thời gian bên trong và thời gian bên ngoài có thể ánh xạ lẫn nhau, một số kĩ thuật trực quan hóa cho dữ liệu tĩnh cũng có thể áp dụng được.
    \item \textit{Dữ liệu động theo thời gian}. Nếu cả thời gian bên trong và bên ngoài đều có nhiều yếu tố thời gian nguyên thủy, thì dữ liệu được xem xét là phụ thuộc vào thời gian. Nói cách khác, dữ liệu chứa các biến phụ thuộc vào thời gian (bên trong) và trạng thái dữ liệu cũng thay đổi theo thời gian (bên ngoài). Thường thì trong trường hợp này, thời gian bên trong và bên ngoài phụ thuộc lẫn nhau và có thể ánh xạ vào nhau. Sự phân biệt rõ ràng giữa thời gian bên trong và bên ngoài thường không được thực hiện bởi các phương pháp trực quan hóa hiện tại bởi xem xét cả 2 chiều thời gian để trực quan hóa là một thách thức.
\end{itemize}