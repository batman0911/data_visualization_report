\subsection{Dữ liệu liên quan và thời gian}
Các khía cạnh liên quan của dữ liệu phụ thuộc vào thời gian đã được kiểm tra rộng rãi trong lĩnh vực cơ sở dữ liệu về thời gian [274,397]. Ở đây, chúng tôi điều chỉnh và phát triển các định nghĩa trong lĩnh vực này. Theo đó, bất kì tập dữ liệu nào cũng có liên quan đến 2 miền thời gian: (1) thời gian bên trong và (2) thời gian bên ngoài. 
\\ \\
\textit{Thời gian bên trong} được coi là chiều thời gian vốn có của mô hình dữ liệu. Thời gian bên trong mô tả thời điểm thông tin tồn tại trong tập dữ liệu là hợp lệ. Ngược lại \textit{thời ian bên ngoài} là thời gian nằm ngoài mô hình dữ liệu. Thời gian bên ngoài là cần thiết để mô tả cách một tập dữ liệu phát triển theo thời gian. Dựa vào số biến thời gian nguyên thủy của thời gian bên trong và thời gian bên ngoài, dữ liệu liên quan đến thời gian có thể được phân loại như sau:
\begin{itemize}
    \item \textit{Dữ liệu tĩnh phi thời gian}. Nếu cả thời gian bên trong lẫn bên ngoài bao gồm một yếu tố thời gian thì dữ liệu hoàn toàn không phụ thuộc vào thời gian. Trong chương này chúng ta sẽ không kiểu dữ liệu đó.
    \item \textit{Dữ liệu thời gian tĩnh}. Nếu thời gian bên trong chứa nhiều hơn một yếu tố thời gian nguyên thủy, trong khi thời gian bên ngoài chỉ chứa một thì dữ liệu có thể được xem là phụ thuộc vào thời gian.
\end{itemize}