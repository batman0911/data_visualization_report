\subsection{Phân loại} \label{sub:3.1.cate}
Để thuận lời cho việc phân loại một cách dễ dàng và giữ cho phân loại các kĩ thuật trực quan hóa đơn giản, chúng tôi tập trung vào những khía cạnh cốt lõi của dữ liệu, thời gian và trực quan hóa. Qua nhiều ví dụ trực quan hóa, chúng tôi sẽ minh họa những khả năng ứng dụng của chúng và những tính năng của các kĩ thuật trực quan hóa này.
\begin{itemize}
    \item \textbf{Dữ liệu} 
    \item[] \begin{itemize}
        \item \textit{Khung tham chiếu} - trừu tượng, không gian
        \item \textit{Các biến} - đơn biến, đa biến
    \end{itemize}
    \item \textbf{Thời gian}
    \item[] \begin{itemize}
        \item \textit{Sắp xếp} - tuyến tính, tuần hoàn
        \item \textit{Thời gian nguyên thủy} - thời điểm, khoảng
    \end{itemize}
    \item \textbf{Trực quan hóa}
    \item[] \begin{itemize}
        \item \textit{Ánh xạ} - tĩnh, động
        \item \textit{Số chiều} - 2D, 3D
    \end{itemize}
\end{itemize}
Để thiết lập những kĩ thuật trực quan hóa khác nhau cho dữ liệu hướng thời gian, chúng ta cần trả lời những câu hỏi sau:
\begin{itemize}
    \item \textit{Cái gì} được biểu diễn ? \textit{Thời gian và dữ liệu}
    \item \textit{Tại sao} nó được biểu diễn ? \textit{Tác vụ của người dùng}
    \item Nó được biểu diễn \textit{thế nào} ? \textit{Biểu diễn trực quan}
\end{itemize}
Trong chương này, chúng tôi chủ yếu nói về thời gian, dữ liệu và biểu diễn trực quan, bỏ qua những tác vụ của người dùng để mọi thứ trở nên dễ hiểu nhất có thể. Các kĩ thuật trực quan hóa đòi hỏi phải hiểu rõ những khâu cụ thể được thực hiện trong quá trình khai phá dữ liệu và trực quan hóa. MacEachen [281] đã đề xuất một phương pháp ở cấp độ thấp giả quyết một phần cụ thể trong miền thời gian. Các khâu này được định nghĩa bởi một tập các câu hỏi quan trọng mà người dùng có thể tìm kiếm câu trả lời bằng biểu diễn trực quan hóa, ví dụ như \textit{sự tồn tại của thành phần dữ liệu}: liệu thành phần này có tồn tại trong một thời điểm (khoảng) nào đó, hoặc là \textit{tốc độ thay đổi}: thành phần dữ liệu này thay đổi nhanh như thế nào theo thời gian. 
\\ \\
Trong phần tiếp theo, chúng tôi sẽ đưa ra các ví dụ cho mỗi loại được nêu trên.
