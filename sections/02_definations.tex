\section{Định nghĩa: đặc tính của dữ liệu hướng thời gian}
Phần này bao gồm các khía cạnh chính để mô tả thời gian và dữ liệu hướng thời gian. Điều quan trọng là phải phân biệt rõ giữa thời gian vật lý và mô hình thời gian trong các hệ thống thông tin. Khi mô hình hóa thời gian trong các hệ thống thông tin, mục tiêu không phải là bắt chước hoàn toàn thời gian vật lý, tuy nhiên để cung cấp mô hình phù hợp nhất để phản ánh các hiện tượng đang được xem xét và hỗ trợ các bài toán phân tích (bằng tay). Hơn nữa, theo Frank [132], không có mô hình đúng duy nhất, có nhiều cách để mô hình hóa thời gian trong các hệ thống thông tin và thời gian cũng được mô hình hóa trong các ứng dụng khác nhau phụ thuộc vào từng bài toán. Các nghiên cứu sâu rộng đã được tiến hành để hình thành khái niệm về thời gian trong nhiều lĩnh vực của khoa học máy tính bảo gồm cả trí tuệ nhân tạo, khai phá dữ liệu, mô phỏng, mô hình, cơ sở dữ liệu ... Chúng tôi phỏng theo các nghiên cứu của Frank [132] và Goralwalla cùng cộng sự [154], trong đó các khía cạnh trực giao chính được trình bày để mô tả các khía cạnh khác nhau của các loại thời gian. Dữ liệu cơ bản được trình bày trong chương 2, chúng tôi tập trung vào các đặc tính của thời gian và dữ liệu hướng thời gian nói riêng. Những khía cạnh này sẽ được mô tả chi tiết sau đây.
\input{sections/02_01_def_char_time.tex}
\subsection{Các đặc tính của dữ liệu hướng thời gian}
Cũng như các tính chất của thời gian, dữ liệu có tác động lớn đến việc thiết kế phương pháp trực quan hóa. Chúng ta cùng lược qua những yếu tố chính của dữ liệu liên quan đến thời gian.
\begin{itemize}
    \item \textit{Quy mô}: \textit{định tính}, \textit{định lượng}. Dữ liệu định lượng được dựa trên một thang đo (rời rạc hoặc liên tục). Dữ liệu định tính mô tả tập hợp các phần tử dữ liệu có thứ thự hoặc không có thứ tự.
    \item \textit{Hệ quy chiếu}: \textit{trừu tượng}, \textit{không gian}. Dữ liệu trừu tượng (ví dụ tài khoản ngân hàng) được thu thập trong bối cảnh phi không gian. Dữ liệu không gian (ví dụ dữ liệu điều tra dân số) chứa thông tin về không gian ví dụ như vị trí địa lý.
    \item \textit{Loại dữ liệu}: \textit{sự kiện}, \textit{trạng thái}. Sự kiện có thể được hiểu là các điểm đánh dấu sự thay đổi trạng thái (ví dụ sự khời hành của đoàn tàu), trong khi đó trạng thái đặc trưng cho các giai đoạn liên tục giữa các sự kiện (ví dụ đoàn tàu đang chạy trên đường sắt).
    \item \textit{Số biến}: \textit{đơn biến}, \textit{đa biến}. Dữ liệu đơn biến chỉ chứa một giá trị dữ liệu tại một thời điểm, trong trường hợp đa biến, dữ liệu tại mỗi thời điểm thể hiện bằng nhiều giá trị dữ liệu.
\end{itemize}
Những phạm trù cơ bản này tạo thành cơ sở cho việc lựa chọn, thiết lập kĩ thuật trực quan hóa phù hợp cho dữ liệu hướng thời gian. 
\subsection{Dữ liệu liên quan và thời gian}
Các khía cạnh liên quan của dữ liệu phụ thuộc vào thời gian đã được kiểm tra rộng rãi trong lĩnh vực cơ sở dữ liệu về thời gian [274,397]. Ở đây, chúng tôi điều chỉnh và phát triển các định nghĩa trong lĩnh vực này. Theo đó, bất kì tập dữ liệu nào cũng có liên quan đến 2 miền thời gian: (1) thời gian bên trong và (2) thời gian bên ngoài. 
\\ \\
\textit{Thời gian bên trong} được coi là chiều thời gian vốn có của mô hình dữ liệu. Thời gian bên trong mô tả thời điểm thông tin tồn tại trong tập dữ liệu là hợp lệ. Ngược lại \textit{thời gian bên ngoài} là thời gian nằm ngoài mô hình dữ liệu. Thời gian bên ngoài là cần thiết để mô tả cách một tập dữ liệu phát triển theo thời gian. Dựa vào số biến thời gian nguyên thủy của thời gian bên trong và thời gian bên ngoài, dữ liệu liên quan đến thời gian có thể được phân loại như sau:
\begin{itemize}
    \item \textit{Dữ liệu tĩnh phi thời gian}. Nếu cả thời gian bên trong lẫn bên ngoài bao gồm một yếu tố thời gian thì dữ liệu hoàn toàn không phụ thuộc vào thời gian. Trong chương này chúng ta sẽ không nghiên cứu kiểu dữ liệu đó.
    \item \textit{Dữ liệu thời gian tĩnh}. Nếu thời gian bên trong chứa nhiều hơn một yếu tố thời gian nguyên thủy, trong khi thời gian bên ngoài chỉ chứa một thì dữ liệu có thể được xem là phụ thuộc vào thời gian. Vì các giá trị được lưu trong dữ liệu phụ thuộc vào thời gian bên trong, dữ liệu thời gian tĩnh có thể được hiểu là một cái nhìn lịch sử về cách một số mô hình xem xét các yếu tố khác nhau của thời gian trong. Chuỗi thời gian chung là một ví dụ tiêu biểu về dữ liệu thời gian tĩnh. 
    \item \textit{Dữ liệu động phi thời gian}. Nếu thời gian bên trong chỉ chứa một mà thời gian bên ngoài chứa nhiều yếu tố thời gian nguyên thủy thì dữ liệu phụ thuộc vào thời gian bên ngoài. Để cho dễ hiểu, dữ liệu thay đổi qua thời gian thì chúng được gọi là động. Vì thời gian bên trong không được xem xét nên chỉ có trạng thái hiện tại của dữ liệu là bảo toàn. Cái nhìn quá khứ không được duy trì ở đây nữa. Có vài kĩ thuật trực quan hóa sẵn có tập trung vào dữ liệu động phi thời gian, tuy nhiên do thời gian bên trong và thời gian bên ngoài có thể ánh xạ lẫn nhau, một số kĩ thuật trực quan hóa cho dữ liệu tĩnh cũng có thể áp dụng được.
    \item \textit{Dữ liệu động theo thời gian}. Nếu cả thời gian bên trong và bên ngoài đều có nhiều yếu tố thời gian nguyên thủy, thì dữ liệu được xem xét là phụ thuộc vào thời gian. Nói cách khác, dữ liệu chứa các biến phụ thuộc vào thời gian (bên trong) và trạng thái dữ liệu cũng thay đổi theo thời gian (bên ngoài). Thường thì trong trường hợp này, thời gian bên trong và bên ngoài phụ thuộc lẫn nhau và có thể ánh xạ vào nhau. Sự phân biệt rõ ràng giữa thời gian bên trong và bên ngoài thường không được thực hiện bởi các phương pháp trực quan hóa hiện tại bởi xem xét cả 2 chiều thời gian để trực quan hóa là một thách thức.
\end{itemize}
\begin{figure}[H] % places figure environment here   
    \centering % Centers Graphic
    \includegraphics[width=0.9\textwidth]{assets/fig_7_2.png} 
    \caption{Các khía cạnh của dữ liệu hướng thời gian} % Creates caption underneath graph
    \label{fig:fig7.2}
\end{figure}