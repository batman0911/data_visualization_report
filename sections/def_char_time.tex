\subsection{Các đặc tính của thời gian}
Các đặc tính của thời gian có thể được chia thành các khía cạnh chung yêu cầu mô hình thời gian đầy đủ cũng như tổ chức phân cấp của thời gian và định nghĩa các yêu tố cụ thể. Các khía cạnh tổng quát bao gồm \textit{thang đo}, \textit{phạm vi}, \textit{sắp xếp} và \textit{góc nhìn}.
\begin{itemize}
    \item \textit{Thang đo}: \textit{thứ tự}, \textit{rời rạc}, \textit{liên tục}. Ở góc độ thứ nhất, chúng tôi xem xét thời gian theo thang đo dọc mà các thành phần của mô hình đã được cho trước. Trong miền thời gian theo \textit{thứ tự}, chỉ có quan hệ thứ tự tương đối được biểu diễn (ví dụ như: trước, sau, trong). Trong miền \textit{rời rạc}, khoảng thời gian cũng được xem xét. Các giá trị thời gian có thể được ánh xạ tới một tập các số nguyên, cho phép lập mô hình định lượng các giá trị thời gian. Miền thời gian rời rạc dựa trên đơn vị (ví dụ: giây, phút). Mô hình thời gian \textit{liên tục} đặc trưng bởi ánh xạ đến các số thực. Nghĩa là giữa hai mốc thời gian, tồn tại một mốc thời gian khác (cũng có thể hiểu là mô hình mật độ thời gian).
    \item \textit{Phạm vi}: dựa trên \textit{thời điểm}, dựa trên \textit{khoảng}. Chúng tôi xem xét phạm vi của các yếu tố cơ bản cấu thành nên miền thời gian. Thời gian dựa trên \textit{thời điểm} có thể được biểu diễn như các điểm Euclide rời rạc trong không gian, tức là có mốc thời gian bằng 0. Như vậy không có thông tin được đưa về khoảng giữa 2 thời điểm. Trái ngược với đó, miền dựa theo \textit{khoảng} có độ lớn lớn hơn 0. Khái niệm này cũng có liên quan chặt chẽ đến khái niệm về độ chi tiết, sẽ được bàn luận sau. Ví dụ, giá trị thời gian ngày 1/5/2014 có thể liên quan đến một thời điểm riêng lẻ là 2014-05-01 00:00:00 là một điểm theo miền dựa trên \textit{thời điểm} cũng có thể là đoạn [2014-05-01 00:00:00, 2014-05-01 23:59:99] trong miền dựa trên \textit{khoảng}
    \item \textit{Sắp xếp}: \textit{tuyến tính}, \textit{tuần hoàn}. Tương tự với nhận thức tự nhiên, chúng ta coi thời gian là một quá trình \textit{tuyến tính} từ quá khứ đến tương lai, tức là mỗi giá trị thời gian có một giá trị duy nhất ở hiện tại cũng như tương lai. Trong cách sắp xếp \textit{tuần hoàn} thời gian được xem xét theo các giá trị định kì (ví dụ mùa trong năm).
    \item \textit{Góc nhìn}: \textit{thứ tự}, \textit{nhánh}, \textit{đa góc nhìn}. Miền thời gian theo \textit{thứ tự} xem xét sự kiện xảy ra sau sự kiện khác. Chi tiết hơn, chúng ta cũng có thể phân biệt giữa thứ tự toàn phần và thứ tự một phần. Trong miền thứ tự toàn phần, chỉ một sự kiện có thể xảy ra trong một thời điểm. Trái với đó, các sự kiện đồng thời hoặc chồng chéo được cho phép trong miền thứ tự một phần. Một dạng phức tạp hơn của mô hình thời gian được gọi là \textit{nhánh} thời gian. Trong mô hình này, nhiều nhánh của thời gian có thể được mô tả với các kịch bản khác nhau (ví dụ mô hình lập kế hoạch). Trái ngược với thời gian phân nhánh, chỉ có một đường thực sự diễn ra theo thời gian, mô hình \textit{đa góc nhìn} cho phép nhiều phương án xảy ra đồng thời theo thời gian. Ví dụ nhiều nhân chứng mô tả cùng một tình huống, mỗi người có một góc nhìn khác nhau và kể một câu chuyện khác nhau.
\end{itemize}
Phân loại thứ bậc thời gian và các yếu tố thời gian cụ thể được xác định dựa trên độ chi tiết, đơn vị thời gian và tính xác định.
\begin{itemize}
    \item \textit{Độ chi tiết và lịch}: \textit{rỗng}, \textit{đơn} và \textit{nhiều}. Để giải quyết độ phức tạp về thời gian và đưa ra các độ chi tiết khác nhau, chúng ta có thể sử dụng vài giải thích vắn tắt hữu ích. Về cơ bản \textit{độ chi tiết} là một khái niệm trừu tượng về thời gian (do chúng ta tự định nghĩa) để hình dung thời gian một cách dễ dàng hơn trong cuộc sống (chẳng hạn như giờ, phút, giây). Tổng quan hơn, độ chi tiết mô tả ánh xạ từ thời gian đến các đơn vị nhỏ hơn hoặc lớn hơn. Nếu độ chi tiết và hệ thống lịch được hỗ trợ bởi mô hình thời gian, chúng tôi mô tả nó như \textit{nhiều} chi tiết. Bên cạnh biến thể phức tạp này, có thể chỉ có một \textit{đơn} chi tiết hoặc là \textit{không} có khái niệm trừu tượng nào trong này được hỗ trợ.
    \item \textit{nguyên thủy thời gian}: \textit{tức thời}, \textit{khoảng thời gian}, \textit{nhịp}. Những nguyên thủy thời gian này có thể được xem như một lớp trung gian giữa các phần tử dữ liệu và miền thời gian. Về cơ bản, thời gian nguyên thủy có thể được chia thành neo nguyên thủy (tuyệt đối) và không neo (tương đối).
\end{itemize}